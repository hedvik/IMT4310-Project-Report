\chapter{Conclusion}
In conclusion, we are very happy with the project which has led to the Conductor Hero game prototype. We believe the game can serve its intended purpose of being an introduction to the world of conducting ensemble orchestras, but that further work will be required for a more complete and immersive experience.

The interdisciplinary nature of the course has also taught us many valuable lessons.
We have learned the importance of working on a tight schedule towards a prototype using a critical path coding method to ensure that we finish on time. This experience will help us be more pragmatic when we meet similar projects in the future. We have gained valuable insights into some of the challenges and gains of working in an interdisciplinary project in terms of communication, methodology, and differences in skill. With this knowledge it should be easier for us to identify and solve areas of friction in future projects, allowing for more effective teamwork. 

For some of us, this project has been a view into the area of leadership and management, sparking an interest in possibilities of pursuing this field as a career. For others, it has been an introduction to the area of game programming, with the unique and varied challenges present within the field. The more experienced members have been presented with the issue of asset and resource pipeline management, ensuring the communication between different applications to realize the groups’ artistic vision. As a group, we have experienced some of the challenges of working in the young field of designing interfaces for VR applications, and how many traditional approaches to design falls short in the VR space. After a successful prototype, we all have an increased interest and confidence in the field of VR technology, which will be invaluable in future VR endeavours. 


