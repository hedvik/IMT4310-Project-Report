\chapter{Introduction}

\section{Background}
Experts in Teamwork (EiT) is a compulsory course for master degree students at NTNU Gjøvik. The main goal is to help the students to develop interdisciplinary teamwork skills~\cite{eit}, such as responsibility for project development, understanding the possibilities which appear from interdisciplinary teams and sharing feedback in a constructive way~\cite{eitQA}. Our village, “Virtual and Augmented Reality for Games, Health and Education”, focuses on how new technology like virtual reality (VR) and augmented reality (AR) can be used not only for entertainment but also in resource demanding sectors like healthcare and education. We were free to come up with our own ideas for this project, as long as it was related to virtual or augmented reality.
\section{Conductor Hero: An introduction to ensemble conducting}
Conductor hero is a VR rhythm game where the player takes on the role of a conductor. Conducting an ensemble throughout a song, the player ensures that the instruments follow the beat of the song and play at the appropriate times. 

The goal of our project was to make an enjoyable game which introduces players to the world of ensemble conducting, an area that can seem quite mysterious to the layman. Through this game, we wanted to expose the player to the domain of ensemble music and encourage them to expand their horizons, build interest and understanding, develop musicality and sensitize the player to musical experiences. Additionally, we are hoping that the game could facilitate the promotion of conducting as a hobby or potentially future choice of education.

The game aims to improve the rhythmic skills of the user through conducting, requiring rhythmic hand movements corresponding to the beats of the music. Additionally cueing helps the user understand how musical tensions are built through different instruments entering and exiting the song in different sections.
VR is a new technology presenting new opportunities and challenges. The gesture tracking that exists in modern VR systems allows for an authentic conductor experience. The immersive nature of VR also allows us to place the player within a fantasy world which strengthens the mood of the music being present. Current room scale solutions are sufficient for this type of game, as we do not require the user to move around.


\section{Market exploration}
Predictably, the number of VR conductor games available at the current time is low. We were not able to find any products directly related to our ideas, however, we looked into similar games in order to see what worked well and what did not.  

\subsection{Audioshield}
Audioshield is one of the more well known VR games on the market. It is a rhythm game where the player blocks incoming spheres in beat with the music. The rhythm of the song determines the patterns and frequencies of the spheres. A sphere is blocked when it hits the player’s shield of the corresponding color. Audioshield served as an inspiration for our placement of the player, and their point of view.

\subsection{Guitar Hero}
While not a VR game, Guitar Hero is a well-known series that covers many of the same aspects as those we want to achieve in our conductor game. In this game the player take on the role as a guitar player in a rock band, and must match the on-screen incoming colored notes. This is done using a plastic guitar controller, and by hitting the notes you score points and engage the audience. While our game focused on conducting rather than guitars, we wanted to create the same feeling of immersion and state of flow as players feel when playing a guitar hero game.

\subsection{Wii music}
Wii music is a console game that allows the player to take different roles in bands and ensembles, including the role of a conductor. Theme-wise, this was the game that came closest to our idea, with similar use of a musical track and beat indication. The game allows the player to mimic a conductor’s movements, but the musical expressiveness is limited. The game has received mixed feedback, and we have found it especially interesting to also look into the aspects that had been poorly rated by players; this way trying to avoid the same pitfalls.