\appendix 
%after this line all chapters will have letters instead of numbers
% spreadsheet data?
\chapter{Appendix} \label{chap:appendix}
\section{Game Design Document}

\subsection{What is “Conductor Hero”?}
Conductor Hero is a VR conducting game that allows the player to take the role of a conductor as the name implies and conduct a band/orchestra of instruments to given music tracks. We have taken inspiration from games like Guitar Hero™ where the focus of the game is not necessarily to be authentic to real life experience, but rather provide an approximation and highlight the experience of conducting instead. 

\subsection{Planned Game Functionality}
Our initial plan for the game includes several different functionalities. First off, we plan to have two different types of game modes. Freestyle mode is where the player can freely cue instruments in and out as they wish where the cueing itself is not scored. Performance mode on the other hand requires the player to cue in instruments at specific points in the music track. In Performance mode we would also like to score the cueing according to how close to correctness it is in relation to timing. 

The band/orchestra will include several instruments. This should include violins, violas, contrabass, harpsichord, glockenspiel and oboes. 

In terms of conducting the player can use their dominant hand to conduct the music with a time signature of 4/4 for the minimum viable product. The other hand can then be used to cue in groups of instruments. The player is scored depending on how close their conducting movements are to being correct and how rhythmically correct it is. We want to be very lenient on the movement itself as it is hard to expect perfect conducting motions every time and rather focus on scoring the rhythmic aspect. At the same time it should not be possible to cheat the conducting pattern. The nondominant hand could also potentially be used to control the intensity of instruments. By doing so we can for example have three different instrument intensities and allow the player to switch between them with the nondominant hand.

Within the game we also want to have a Conductor Table/Dashboard that contains all relevant UI elements. The list of what we want to display on this screen is as follows: Score, communication on where to cue instruments in, duration left on song and current song volume. 

\subsection{Project Scope} 
For this project, we plan to scope for a minimum viable product containing only one level as we have a limited amount of development time. If the MVP milestone is met early then we can also potentially add the ability to share the player’s performance from freestyle mode as a performance mode challenge. 
