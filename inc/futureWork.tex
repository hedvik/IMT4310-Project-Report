\chapter{Future Work}

While we managed to get further than we had expected, there is still a lot of work left unfinished. Time constraints was a challenge throughout the project, and we believe that the finished product could be significantly improved with further testing. With more time available, user testing and user experience (UX) design would have been a priority for the group.  User feedback can be an invaluable resource during the development of a product, and we believe that it would be beneficial to make this a priority in the possible further development of Conductor Hero. 

Turning the prototype into a fully fledged game would require a complete rewrite of most of the game systems, to create more generalized and reusable solutions. However, this rewrite would be necessary to achieve other desirable features, like more levels, songs, and the potential for user-generated content. A fully fledged game would also require us to design a complete user interface for the user to interact with, which would present us with more challenges in the VR space. 

During the initial design phase, we came up with several ideas for extending the mechanics of the game, for example, a level where you don’t just conduct the instruments but also the virtual environment, creating a visual experience that reinforced the feeling of the music. To keep the scope and mechanics at a manageable level these ideas were not pursued, however, we would like to explore these ideas in an extended project.
